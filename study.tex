\section{Study Results}\label{LPA}

\subsection{Objectives}
In order to assess the feasibility of extending the KCFN to the area surrounding
Licoln Prep, it is essential to establish well-defined targets for any potential
work. The proposal that follows is intended to satisfy the following parameters:

\begin{description}
\item[Purpose] This proposal is for a pilot project intended to showcase how
free networks can be used to increase the state of connectivity in Kansas City's
urban core. Of particular interest is how those KCMSD students without in-home
Internet access can brought online. While it is meant to serve as a potential foundation for future work, it
should be capable of producing a sustainable positive impact on the
neighborhood in and of itself.

\item[Scope] The study area is bounded by Truman Road on the north, Prospect
Avenue on the east, $27^{th}$ Street on the south, and 71 Highway \& Troost
Avenue on the west, for a total area of 1.035$mi^{2}$. In addition to involving
the School District, the project should involve an array of area residents,
businesses, non-profits, and community groups. 

\item[Coverage] While the entire study area should be taken into account,
residential access is the highest priority. At the same time, in-home coverage
cannot be achieved without the involvement and investment of the community.
Therefore, our goal should be to enable any and all blocks within
the geographic scope of this effort to participate.
-Figure: population density map.
As illustrated in the figure above, those areas south and east of Lincoln Prep
should receive special attention, as well as Parade Park Homes, at the northern
edge of the coverage area.

\item[Functionality] Those that elect to participate in the network, in addition
to gaining access to resouces published on the KCFN, should have the
ability to purchase low-cost Internet access. KCMSD students and faculty that live in
the area should be able to access the KCMSD network via a secure, authenticated portal,
and should be able to utilize the KCMSD's Internet connection.

\item[Performance] While exact performance figures will depend case-by-case on a
number of factors, the KCFN should enable broadband connectivity capable of supporting
telephony, web 2.0, and multimedia applications.

\item[Cost] The total cost of
accessing the Internet via the KCFN, including hardware, should be lower than existing
alternatives over the course of one year.

\item[Sustainability] Above all, this effort should aim to foster a digital commons 
that is sustainable in the long term --- focusing first and foremost on education, grassroots
support, and the capacity for ongoing, organic grwoth.
\end{description}


\subsection{Survey Information}
The primary physical considerations in determining build feasibility and an
appropriate course of action are topographic terrain and RF environment. We
surveyed the target area in late August of 2013, analyzing the lay of the land
and taking ground-level RF spot samples: \par
-figure: topo map // \par
The terrain in question, while not without its challenges, is actually quite
suited for a wireless networking project. Lincoln Prep's commanding vista covers
a great many points north and west, in addition to allowing for redundant
interconnection to existing KCFN infrastructure.  Those areas occluded by the
gradients that lead to 71 Highway and the $18^{th}$ \& Vine valley are
significantly devoid of residences.  Close proximity to
existing KCFN infrastructure at $18^{th}$ \& Vine compensates adequately for the
lack of clear line of sight  between Licon Prep and the Parade Park Homes
residential area. The areas to the south and east of Lincoln Prep contain the
majority of residences in the target area. Light to moderate foliage and slight
gradients there preclude one-hop access 
-figure: rf heatmaps/charts //
-rf environment //


\subsection{Findings}
-list of possible tower sites ranked by utility estimation
-general cost estimate
-specific requirements for initial tower
-number of participants per node/relay/tower
-number of nodes/relays/towers per area
-overview of how lincoln network would be built, general size, very loose
estimates of performance. 

\subsection{Proposal}

risk 1-. to avoid the risks detected\\
-- ownership to the end user (following the current approach)\\

