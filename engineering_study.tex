\title{LPA Cloud Engineering Study}
\author{
        Pau Esrich \\
                qmp developer\\
        	guifi.net foundation\\
        UPC, Barclenoa, \underline{Catalonia}
            \and
        Roger Vinas\\
        	qmp developer\\
        	guifi.net foundation\\
        UPC, Barclenoa, \underline{Catalonia}
            \and
	Isaac Wilder \\
	    \and
	Rick Dean
}
\date{\today}

\documentclass[12pt]{article}

\begin{document}
\maketitle

\begin{abstract}
This is the paper's abstract \ldots
\end{abstract}

\section{Introduction}\label{Introduction}
\subsection{The Network Commons or Free Vs. Proprietary Networks}
-A local/metropolitan network
-ability to interconnect with other networks
-ability to interconnect with global networks
-ability to interconnect with other local networks
-ability to offer services
-ability to tunnel.
-No local transit costs due to stakeholdership

\subsection{Wireless Communications}
-Inexpensive
-Lossy
-Licensed/unlicensed bands
--microwave
--higher frequencies


\subsection{Kansas City Freedom Network}\label{KCFN}
\subsubsection{Architecture}
-Mesh Networks
-The Stack: Link,Tower,Relay,Node
--layer 0 tech:mounts,masts,penetrations,power,utility rooms,etc
-The firmware: ip, routing, administration, os

\subsubsection{Existing Infrastructure}
-include figures

\section{Study Results}\label{LPA}
\subsection{Requirements}
-scope -coverage -performance -cost -reliability
-two ptp links to existing infrastructure
-one square mile
-best effort coverage
-reasonable expectation of 3mbps
\subsection{Survey Information}
-terrain
-rf environment
\subsection{Findings}
-list of possible tower sites ranked by utility estimation
-general cost estimate
-specific requirements for initial tower
-number of participants per node/relay/tower
-number of nodes/relays/towers per area
-variables/risks
-overview of how lincoln network would be built, general size, very loose
estimates of performance. 

\section{
\section{Conclusion}


\end{document}
